%%%%%%%%%%%%%%%%%%%%%%%%%%%%%%%%%%%%%%%%%
% Twenty Seconds Resume/CV
% LaTeX Template
% Version 1.1 (8/1/17)
%
% This template has been downloaded from:
% http://www.LaTeXTemplates.com
%
% Original author:
% Carmine Spagnuolo (cspagnuolo@unisa.it) with major modifications by 
% Vel (vel@LaTeXTemplates.com)
%

% License:
% The MIT License (see included LICENSE file)
%
%%%%%%%%%%%%%%%%%%%%%%%%%%%%%%%%%%%%%%%%%

%----------------------------------------------------------------------------------------
%	PACKAGES AND OTHER DOCUMENT CONFIGURATIONS
%----------------------------------------------------------------------------------------

\documentclass[letterpaper]{twentysecondcv} % a4paper for A4


%----------------------------------------------------------------------------------------
%	 PERSONAL INFORMATION
%----------------------------------------------------------------------------------------

% If you don't need one or more of the below, just remove the content leaving the command, e.g. \cvnumberphone{}

\profilepic{lc.jpeg} % Profile picture

\cvname{Leandro E. \\Colombo Viña} % Your name
\cvjobtitle{Profesor \& Emprendedor} % Job title/career

\cvdate{9 Octubre 1981} % Date of birth
\cvaddress{Larrea 910 8\textdegree D, CABA} % Short address/location, use \newline if more than 1 line is required
\cvnumberphone{+54 9 11 3001-5328} % Phone number
\cvsite{https://leo.bitson.group} % Personal website
\cvmail{colomboleandro@ifts18.edu.ar} % Email address

%----------------------------------------------------------------------------------------

\begin{document}

%----------------------------------------------------------------------------------------
%	 ABOUT ME
%----------------------------------------------------------------------------------------

\aboutme{Desde el 2004 trabajo en educación. Preceptor, maestro de enseñanza práctica y en Formación Profesional. Desde entonces soy un educador comprometido con el mundo del trabajo. Actualmente la educación superior es donde hoy disfruto mi labor docente. Desde el 2011 estoy en el sector productivo en el mundo IT. Siempre impulsando y trabajando con Software Libre. En 2014 fundamos la Cooperativa de Trabajo BITSON Ltda. que brinda desarrollos de software y hardware utilizando tecnologías libres.} % To have no About Me section, just remove all the text and leave \aboutme{}

%----------------------------------------------------------------------------------------
%	 SKILLS
%----------------------------------------------------------------------------------------

% Skill bar section, each skill must have a value between 0 an 6 (float)
\skills{{Exigencia/5},{Proactividad/4.8},{Responsabilidad/5.7},{Management/5},{Análisis/4.2},{Docencia/4.7}}

%------------------------------------------------

% Skill text section, each skill must have a value between 0 an 6
\skillstext{{DevOps/4},{Python/5},{Backend/4.5},{Frontend/2.3}}

%----------------------------------------------------------------------------------------

\makeprofile % Print the sidebar

%----------------------------------------------------------------------------------------
%	 INTERESTS
%----------------------------------------------------------------------------------------

\section{Intereses}

Soy un apasionado por el conocimiento y la tecnología. Me encanta explorar nuevas formas de trabajo y compartir nuevos saberes,
aplicando la tecnología a la enseñaza y al mundo del trabajo.

%----------------------------------------------------------------------------------------
%	 EDUCATION
%----------------------------------------------------------------------------------------

\section{Educación}

\begin{twenty} % Environment for a list with descriptions
	\twentyitem{2018-hoy}{Mg. {\normalfont Gestión Estratégica de Sist. y Tec. de la Información}}{FCE-UBA}{\footnotesize \textit{en curso}}
	\twentyitem{2014-2015}{Prof. Disciplinas Industriales}{INSPT-UTN}{\footnotesize Nivel medio y uperior no universitario.}
	\twentyitem{2011-2014}{Técnico Superior en Informática Aplicada}{INSPT-UTN}{\footnotesize Promedio General con y sin aplazos: 9,29\\Abanderado Insignia Nacional 2013}
	\twentyitem{2008-2008}{Instructor de Formación Profesional}{CFP 34}{\footnotesize Formación técnica y pedagógica.}
	\twentyitem{1995-2000}{Técnico en Electrónica}{Pío IX (A-66)}{\footnotesize Título secundario.}
	%\twentyitem{<dates>}{<title>}{<location>}{<description>}
\end{twenty}

%----------------------------------------------------------------------------------------
%	 PUBLICATIONS
%----------------------------------------------------------------------------------------

\section{Cursos y Certificaciones}

\begin{twentyshort} % Environment for a short list with no descriptions
	\twentyitemshort{2016}{Scrum Grand Master. {\scriptsize SCEU-FRBA-UTN}}
	\twentyitemshort{2015}{Operador en Comunicación Social y Digital. {\scriptsize FUNDESCO, Ministerio de Trabajo}}
	\twentyitemshort{2015}{Teletrabajador Genérico. {\scriptsize FUNDESCO, Ministerio de Trabajo}}
	\twentyitemshort{2014}{Linux Certified Security System. {\scriptsize Carrera Linux Argentina}}
	\twentyitemshort{2013}{Carrera C Expert. {\scriptsize Carrera Linux Argentina}}
	\twentyitemshort{2012}{Programa Ejecutivo en Project Management. {\scriptsize SCEU-FRBA-UTN}}
	\twentyitemshort{2011}{Linux System Administrator Expert. {\scriptsize Carrera Linux Argentina}}
	\twentyitemshort{2008}{Elaboración de Productos Multimediales. {\scriptsize CFP 34}}
	%\twentyitemshort{<dates>}{<title/description>}
\end{twentyshort}

%----------------------------------------------------------------------------------------
%	 AWARDS
%----------------------------------------------------------------------------------------

%\section{Awards}
%
%\begin{twentyshort} % Environment for a short list with no descriptions
%	\twentyitemshort{1987}{All-Time Best Fantasy Novel.}
%	\twentyitemshort{1998}{All-Time Best Fantasy Novel before 1990.}
%	%\twentyitemshort{<dates>}{<title/description>}
%\end{twentyshort}

%----------------------------------------------------------------------------------------
%	 EXPERIENCE
%----------------------------------------------------------------------------------------

\section{Experiencia Laboral}

\begin{twenty} % Environment for a list with descriptions
	\twentyitem{2014-hoy}{Presidente, Socio fundador.}{BITSON}{
		Desarrollo de software y hardware. \url{https://bitson.group/}}
	\twentyitem{2010-hoy}{Profesor Titular}{IFTS18}{Materias de la carrera de Técnico Superior en Análisis de Sistemas.}
	\twentyitem{2011-2014}{Consultor}{freelance}{Cableado estructurado.
		Administración GNU/Linux. Capacitaciones Empresariales.}
	\twentyitem{2007-2015}{Instructor}{CFP34}{Técnicas de Programación (Python y Software Libre). Operador Básico GNU/Linux y Administración de Redes en GNU/Linux. Recursos Educativos 2.0. Informática Básica.}
	\twentyitem{2010-2014}{Community Manager}{CFP34}{Administración de Google Apps for Educations y perfiles en redes sociales.}
	\twentyitem{2004-2014}{SysAdmin / Maestro de Enseñanza Práctica / Preceptor}{Pío IX}{}
	\twentyitem{2000-2004}{Técnico de Laboratorio}{Celtronics SRL/IMD SRL}{}
	%\twentyitem{<dates>}{<title>}{<location>}{<description>}
\end{twenty}

%----------------------------------------------------------------------------------------
%	 OTHER INFORMATION
%----------------------------------------------------------------------------------------

\section{Otras actividades}

\begin{twentyshort} % Environment for a short list with no descriptions
   	\twentyitemshort{2018-hoy}{Python Software Foundation, Managing Member.}
	\twentyitemshort{2016-hoy}{Organizador Conferencia Anual de Python Argentina.}
	\twentyitemshort{2015-hoy}{Socio Fundador y Tesorero Asociación Civil Python Argentina.}
	\twentyitemshort{2014/15/17}{Disertante FLISOL \url{http://bitson.group/slides}}
	\twentyitemshort{2014}{Administrador del Centro de Salesianos Cooperadores Pío IX.}
	\twentyitemshort{2010-2012}{Jornadas de Tecnologías Educativas}
	\twentyitemshort{2011}{Trayecto Formación Contínua Educación Técnica Profesional - DGEGP}
	\twentyitemshort{2008}{Fotografía Reflex Digital, Edición de Imágenes.}
	%\twentyitemshort{<dates>}{<title/description>}
\end{twentyshort}

%\subsection{Review}
%
%Alice approaches Wonderland as an anthropologist, but maintains a strong sense of noblesse oblige that comes with her class status. She has confidence in her social position, education, and the Victorian virtue of good manners. Alice has a feeling of entitlement, particularly when comparing herself to Mabel, whom she declares has a ``poky little house," and no toys. Additionally, she flaunts her limited information base with anyone who will listen and becomes increasingly obsessed with the importance of good manners as she deals with the rude creatures of Wonderland. Alice maintains a superior attitude and behaves with solicitous indulgence toward those she believes are less privileged.

%----------------------------------------------------------------------------------------
%	 SECOND PAGE EXAMPLE
%----------------------------------------------------------------------------------------

%\newpage % Start a new page

%\makeprofile % Print the sidebar

%\section{Other information}

%\subsection{Review}

%Alice approaches Wonderland as an anthropologist, but maintains a strong sense of noblesse oblige that comes with her class status. She has confidence in her social position, education, and the Victorian virtue of good manners. Alice has a feeling of entitlement, particularly when comparing herself to Mabel, whom she declares has a ``poky little house," and no toys. Additionally, she flaunts her limited information base with anyone who will listen and becomes increasingly obsessed with the importance of good manners as she deals with the rude creatures of Wonderland. Alice maintains a superior attitude and behaves with solicitous indulgence toward those she believes are less privileged.

%\section{Other information}

%\subsection{Review}

%Alice approaches Wonderland as an anthropologist, but maintains a strong sense of noblesse oblige that comes with her class status. She has confidence in her social position, education, and the Victorian virtue of good manners. Alice has a feeling of entitlement, particularly when comparing herself to Mabel, whom she declares has a ``poky little house," and no toys. Additionally, she flaunts her limited information base with anyone who will listen and becomes increasingly obsessed with the importance of good manners as she deals with the rude creatures of Wonderland. Alice maintains a superior attitude and behaves with solicitous indulgence toward those she believes are less privileged.

%----------------------------------------------------------------------------------------

\end{document} 
